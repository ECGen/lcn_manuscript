
\textbf{add newer lit studies here.}
\begin{itemize}
\item Daves 2016 paper
\item Jamie's 2016 paper
\item McKnight's paper: ecological networks are connected across
  aquatic and terrestrial systems and impact evolutionary processes
\item Nat Eco Evo 2017:
\item Liza and Amy 2017: 
\item Ghering 2017: Tree genetics defines fungal partner communities
  that may confer drought tolerance
\item Bothwell 2017: Precipitation gradients and river networks drive
  genetic connectivity and diversity in a foundation riparian tree
\item Poisot
\end{itemize}

Genetic variation of a foundation rockweed species affects associated
communities Jormalainen et al September 2017 Ecology

Andreazzi, C.S, J. N. Thompson, and P. R. Guimarães, Jr. 2017. Network
structure and selection asymmetry drive coevolution in species-rich
antagonistic interactions. American Naturalist 190:99-115

Toju, H. M. Yamamichi, P. R. Guimarães, Jr., J. M. Olesen, A. Mougi,
T. Yoshida, and J. N. Thompson. 2017. Species-rich networks and
eco-evolutionary synthesis at the metacommunity level. Nature Ecology
and Evolution 1:0024. DOI: 10.1038/s41559-016-0024

Dáttilo, W., N. Lara-Rodriguez, P. Jordano, P. R. Guimarães, Jr.,
J. N. Thompson, R. J. Marquis, L. P. Medeiros, R. Ortiz-Pulido,
M. A. Marcos-García, and V. Rico-Gray. 2016. Unraveling Darwin's
entangled bank: architecture and robustness of mutualistic networks
with multiple interaction types. Proceedings of the Royal Society B
283: 20161564.

Toju, H., P. R. Guimarães, Jr., J. M. Olesen, and
J. N. Thompson. 2015. Plant communities and below-ground plant-fungal
networks. Sciences Advances 1:e1500291

Toju, H., P. R. Guimarães, Jr., J. M. Olesen, and J. N. Thompson. 2014
Assembly of complex plant-fungal networks. Nature Communications
5:5273 DOI:10.1038/ncomms6273

Guimarães, P. R., Jr., P. Jordano, and J. N. Thompson. 2011. Evolution
and coevolution in mutualistic networks. Ecology Letters 14:877-885.

Cuautle, M., and J. N. Thompson. 2010. Evaluating the co-polliinator
network structure of two sympatric Lithophragma species with different
morphology. Oecologia 162:71-80.

Díaz-Castelazo, C., P. R. Guimarães, Jr., P. Jordano, J. N. Thompson,
R. J. Marquis, and V. Rico-Gray. 2010. Changes of a mutualistic
network over time: reanalysis over a 10-year period. Ecology
91:793-801.

Guimarães, P. R., Jr., V. Rico-Gray, P.S. Oliveira, T. J. Izzo,
S. F. dos Reis, and J. N. Thompson. 2007. Interaction intimacy affects
structure and coevolutionary dynamics in mutualistic networks. Current
Biology 17:1797-1803

Guimarães, P.R., V. Rico-Gray, S.F. dos Reis and J.N. Thompson
(2006). Asymmetries in specialization in ant–plant mutualistic
networks. Proc. R. Soc. B 273: 2041–2047.

Keith, A.R., J.K. Bailey, M.K. Lau, and T.G. Whitham.  2017.
Genetics-based interactions of foundation species affect community
diversity, stability, and network structure.  Proceedings of the Royal
Society B 284: 20162703. http://dx.doi.org/10.1098/rspb.2016.2703.

Lamit, L.J., P.E. Busby, M.K. Lau, Z.G. Compson, T. Wojtowicz,
A.R. Keith, M.S. Zinkgraf, J.A. Schweitzer, S.M. Shuster,
C.A. Gehring, and T.G. Whitham.  2015.  Tree genotype mediates
covariance among diverse communities from microbes to arthropods.
Journal of Ecology 103:840–850.

Lau, M.K., A.R. Keith, S.R. Borrett, S.M. Shuster, and T.G. Whitham.
2016.  Genotypic variation in foundation species generates network
structure that may drive community dynamics and evolution.  Ecology
97:733-742.

Crutsinger, G.M., 2016 A community genetics perspective: opportunities
for the coming decade. New Phytologist, 210, 65-70.

