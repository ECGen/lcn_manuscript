% This file was converted to LaTeX by Writer2LaTeX ver. 1.4
% see http://writer2latex.sourceforge.net for more info
\documentclass{article}
\usepackage[ascii]{inputenc}
\usepackage[T1]{fontenc}
\usepackage[english]{babel}
\usepackage{amsmath}
\usepackage{amssymb,amsfonts,textcomp}
\usepackage{array}
\usepackage{hhline}
\title{}
\author{Lau, Matthew K.}
\date{2020-10-12}
\begin{document}
Genotypic variation in a foundation tree results in heritable ecological network structure of an associated community

Biological evolution occurs in the context of complex ecosystems of
interacting species whereby natural selection defines the structure of
ecological networks. Fundamental to understanding evolutionary
processes is elucidating the genetic basis to ecological network
structure, which is defined by interactions among species. Although
previous work has demonstrated that genotypic variation in foundation
species contributes to interaction network structure, we are not aware
of a study that has quantified the genetic contribution to network
structure or shown network structure to be a heritable trait. To
examine this, in a %
%Yes, I agree. \ By we, I didn{}'t mean just us, so the rephrasing is good.
%Tom Whitham
%October 9, 2020, 10:40 AM
%
%This sounded as though the garden was setup just for this study. Though I would love to take credit for {\textquotedbl}establishing{\textquotedbl} them, I thought it best to re{}-write this a bit in case Kevin Floate ever reads this. 
%Lau, Matthew K.
%October 7, 2020, 4:45 PM
20+ year common garden we observed interactions among nine epiphytic
lichen species associated with genotypes of (Populus angustifolia), a
foundation species of riparian ecosystems. We constructed signed,
weighted, directed interaction networks for the lichens and conducted
genetic analyses of whole network similarity, degree and
centralization. We found three primary results. First, using multiple
metrics, tree genotype significantly predicted lichen network
structure; i.e., clonal replicates of the same genotype tended to
support more similar lichen networks than different genotypes. Second,
broad sense heritability estimates show that plant genotype explains
network similarity (H2 = 0.41), network degree (H2 = 0.32) and network
centralization (H2 = 0.33). Third, one of the examined tree traits,
bark roughness, was also heritable (H2 = 0.32) and significantly
correlated with lichen network similarity (R2 = 0.26), supporting a
mechanistic pathway from variation in a heritable tree trait and the
genetically based variation in lichen network structure that selection
can act upon. We conclude that tree genotype can influence not only
the relative abundances of organisms but also the interaction network
structure of associated organisms. Given that variation in network
structure can have consequences for the dynamics of communities
through altering system-wide stability and resilience and modulating
perturbations, these results have important implications for the
evolutionary dynamics of ecosystems.

networks {\textbar} heritability {\textbar} community {\textbar}
genetics {\textbar} lichen {\textbar} cottonwood {\textbar} Populus {\textbar} common garden 

Significance Statement Evolution occurs in the context of ecosystems comprised of complex ecological networks. Research
at the interface of ecology and evolution has primarily focused on pairwise interactions among species and have rarely
included a genetic component to network structure. Here, we used a 20+ year common garden experiment to reveal the
effect that genotypic variation can have on networks of lichens that colonize the bark of a foundation tree species. We
found that lichen interaction network structure is genetically based and primarily driven by bark roughness. These
findings demonstrate the importance of genetic variation and evolutionary dynamics in shaping ecological networks as
evolved traits. In particular, this study points to the importance of assessing the effect of foundation species
genetics on the structure of species interactions that can generate heritable network variation that selection can act
upon. 

M.L. and L.L. conceived the study, M.L. and L.L. conducted the field work, R.N. assisted in lichen identifications, M.L.
wrote the first draft of the manuscript, S.B. and T.W. contributed substantively to the conceptual development, T.W.
established the common garden. All authors contributed to revisions of the manuscript. The authors have no conflicts of
interest. 1Dr. Matthew K. Lau. E-mail: matthewklau@fas.harvard.edu www.pnas.org/cgi/doi/10.1073/pnas.XXXXXXXXXX PNAS
{\textbar} September 23, 2020 {\textbar} vol. XXX {\textbar} no. XX {\textbar} 1--20 DRAFT 

Evolution occurs in the context of complex ecological networks. Community genetics studies have shown that genetic
variation in foundation species, which have large effects on communities and ecosystems by modulating and stabilizing
local conditions (1), plays a significant role in defining distinct communities of interacting organisms: such as,
endophytes, pathogens, lichens, arthropods, and soil microbes (2--4). Multiple studies have now demonstrated that
genetic variation influences numerous functional traits (e.g., phytochemical, phenological, morphological) that in
combination results in a multivariate functional trait phenotype (5) in which individual plant genotypes support
different communities and ecosystem processes (6, 48). The importance of genetic variation in structuring ecological
systems was recently reviewed (7), and not only were many instances of strong genetic effects found in many ecosystems
but the effect of intraspecific variation was at times greater than inter-specific variation. There is now evidence to
support that selection, acting on this heritable variation, tends to occur among groups of species (8) and that genetic
variation and phylogenetic relatedness contribute to variation in community assembly (9) and species interactions (6,
10, 11), which shape the structure of ecological interaction networks (12--14). 

In this community-level evolutionary context, the ``genetic similarity rule'' provides a useful framework for
approaching the nexus of evolutionary and community dynamics in the context of complex interaction networks. In a study
combining experimental common gardens and landscape-scale observations of interactions between Populus spp.
(cottonwoods) and arthropods, (15) observed that individuals genotypes that are more genetically similar will tend to
have similar phytochemical traits and thus tend to have similar interactions with other species than individuals that
are less similar. However, studies in the network ecology literature generally do not include a genetic component (16)
and community genetics studies have primarily focused on community composition in terms of the abundance of species
(7). Some studies have examined the effects of genetic variation on trophic chains in plant-associated communities
(including Populus, Solidago, Oenothera, Salix) (17--21) and generally found that increasing genotypic diversity leads
to increased trophic complexity. Only two other studies, that we are aware of, have explicitly examined the effect of
genotypic variation on the structure of interaction networks between tree individuals and associated herbivores (22,
23) and both found that genotypic diversity generates increased network modularity (i.e., compartmentalization).
However, both of these studies were examining networks at the scale of forest stands, rather than networks associated
with %
%Yes, you{}'re right in saying that in Lau et al 2016, there is no network replication and thus no way of caclculating heritabilty. Fig 1B \ is community composition on genotypes (see the legend). The take{}-home was that the variation in community composition among genotypes contributes to the structure of interactions among genotypes and the community, hence Fig 2. 
%
%Tom {}-- I still have questions about Fig. 2. \ It shows differences among tree genotypes at the stand level that differ from a random model. \ But without replication, you can{}'t say if they differ statistically from one another in centralization, modularity, and nestedness?
%
%And also yes, per your second point. The checkerboard allows us to construct networks on each tree. And thus, like other heritabiliity estimates, there is variation within and among genotypes that allows us \ to calculate heritability. 
%
%I agree that this does seem like the major point of this paper and the most novel. I think that the Zatinska paper might have a similar estimate of heritability of food{}-webs in epiphytic pitcher plants, but I need to go back and read that again. 
%
%I don{}'t recall any other papers calculating heritability outside our group. \ They could have, but apparently didn{}'t think heritability was an important metric to calculate or didn{}'t understand how to calculate it from Steve{}'s 2006 paper. Steve had a heritability calculator that he made available to the group, but he didn{}'t publish it, probably assuming it was obvious from his equations? \ Interesting to see if Zatinska did calculate it and perhaps others?
%Lau, Matthew K.
%October 7, 2020, 4:18 PM
%
%Just to make sure I understand. \ Even though Fig 1B from Lau et al. 2016 shows standard deviation bars around each of 10 genotypes (4 reps each), this did not translate to replication in the network analysis (Fig. 2), i.e., the reps were pooled to get a combined network so that heritability could not be calculated. \ Is this right? \ In the current study, the checkerboard square of 100 cells gives the replication needed to calculate heritability that couldn{}'t be done before. If I still don{}'t have it, I think a call is needed as this is an important point I need to understand and it needs to be clearer to the readers as well. \ I think I was stuck on having replication in terms of genotypes and their communities, but not having replication in terms of networks.
%Tom Whitham
%October 6, 2020, 4:00 PM
%
%Again, the network was at the stand scale (among trees), rather than at the scale of the tree. I.e., there was no replication of networks. This is why we haven{}'t been able to get an estimate of network heritability in the past. I{}'ve repeated this a lot, so if you still aren{}'t getting it, we should have a phone call to explain it more clearly, as this is a key novelty of the paper.
%
%
%Lau, Matthew K.
%October 5, 2020, 3:24 PM
%
%Previous sentence says individual trees for the Keith paper and here it says forest stands?? Confusing. \ Keith{}'s paper had 4 reps of each of 10 genotypes in a common garden
%Tom Whitham
%September 28, 2020, 12:37 PM
individual trees; therefore, neither was able to observe replicated networks in order to statistically test for genetic
effects on network structure and quantify the genetic component (i.e. heritable variation) in network structure. 

Network theory and evidence from empirical studies in ecology have demonstrated that indirect effects can lead to
self-organization, producing sign-changing, amplifying and/or dampening effects (24, 25). The development of
interspecific indirect genetic effects (IIGE) theory (26) in evolutionary biology points to the importance of studying
the genetic basis of interaction network structure because genetic based differences in network structure among
individuals can be acted upon by natural selection when there are fitness consequences of different networks of IIGEs
that can result in community evolution (Whitham et al. %
%Whitham, T.G., G.J. Allan, H.F. Cooper, and S.M. Shuster. \ 2020. \ Intraspecific genetic variation and species interactions contribute to community evolution. \ Annual Review of Ecology, Evolution and Systematics 51:587{}-612.
%Tom Whitham
%September 28, 2020, 2:11 PM
2020). For example, although the analysis was of abundances rather than interaction networks, Gehring et al. (%
%Gehring, C.A., D. Flores{}-Renter\'ia, C.M. Sthultz, T.M. Leonard, L. Flores{}-Renter\'ia, A.V. Whipple, and T.G. Whitham. \ 2014. \ Plant genetics and interspecific competitive interactions determine ectomycorrhizal fungal community responses to climate change. \ MOLECULAR ECOLOGY 23:1379{}--1391.
%Tom Whitham
%September 28, 2020, 1:43 PM
2014, %
%Gehring, C.A., C.M. Sthultz, L.H. Flores{}-Renter\'ia, A.V. Whipple, and T.G. Whitham. \ 2017. \ Tree genetics defines fungal partner communities that may confer drought tolerance. \ Proceedings of the National Academy of Sciences 114:11169{}-11174.
%
%Tom Whitham
%September 28, 2020, 1:43 PM
2017) found that the mycorrhizal communities on the roots of drought tolerant and intolerant trees are dominated by
different orders of ectomycorrhizal fungal mutualists that also differ in the benefits they provide that enhance tree
performance. Because drought tolerant genotypes are 3x more likely to survive record droughts, selection acts both on
the tree and its fungal community and with increased drought the community phenotype has changed over time. Also, in an
antagonistic interaction context, Busby et al. (%
%Busby, P.E., L.J. Lamit, A.R. Keith, G. Newcombe, C.A. Gehring, T.G. Whitham, and R. Dirzo. \ 2015. \ Genetics{}-based interactions among plants, pathogens and herbivores define arthropod community structure. \ Ecology 96:1974{}--1984.
%Tom Whitham
%September 28, 2020, 5:11 PM
2015) found that with the addition of a damaging leaf pathogen to cottonwoods in a common garden, the impacts of these
strong interactors results in a different and diminished community of arthropods relative to control trees. Thus,
selection acting on the tree may alter the network structure of associated communities in which different networks of
communities are most likely to survive pathogen outbreaks. Regardless of whether the IIGE is unilateral (i.e., tree
affects the community) or reciprocal (i.e., the community also affects the relative fitness of the tree), selection on
tree, community or both can change network structure (Whitham et al. 2020) and thereby alter community dynamics.
Evolutionary applications of network theory have demonstrated that indirect effects of interactions among species can
lead to network structures %
%I think it reads fine as it is now written. \ Is there another meaning that needs to be clarified?
%Tom Whitham
%October 6, 2020, 4:36 PM
%
%So, should the {\textquotedbl}can{\textquotedbl} in {\textquotedbl}among species can lead to network{\textquotedbl} be removed?
%Lau, Matthew K.
%October 2, 2020, 7:32 PM
%
%Don{}'t need to qualify with {}``can{}'' twice.
%Tom Whitham
%September 28, 2020, 2:20 PM
that amplify or dampen the effects of selection (27). For example, networks that form a star-like structure in which
there is a central species or core group of species that interact with other, peripheral species, can amplify selection
events. Empirically, network analysis of the structure of bipartite (i.e., two-mode) mutualistic networks has shown in
multiple cases that nestedness, or the degree to which species tend to interact with similar subsets of the community,
tends to promote stability and resilience to disturbances (28) NEED TO ADD %
%I added a sentence here that I think is a stronger ending for this paragraph and a good setup for a key discussion point that I think we could focus on more. 
%Lau, Matthew K.
%October 7, 2020, 4:31 PM
BASCOMPTE2014. As such differences in network structure could occur without observable differences in species richness
or community composition, which have been the primary focus of almost all previous community genetics studies. Thus, it
is important to quantify how network structure changes in response to genetic variation in order to fully understand
evolutionary dynamics in complex communities. 

Here, we investigate how genetic variation in a foundation tree species determines the structure of a network of
interactions among a community of tree associated lichen species. Previous studies have examined aspects of networks
(29). Here we examine the genetic basis of network structure on a community of sessile lignicolous (i.e., bark) lichens
on cottonwood trees. Using a long-term (20+ years), common garden experiment with clonally replicated Populus
angustifolia individuals of known genetic identity . We focused on a community of 9 epiphytic lichen species, as
previous research has demonstrated significant compositional responses of epiphytes to genotypic variation (30, 31). In
addition, the life-history characteristics of lichens, having highly localized, direct contact interactions and slow
population turnover rates, facilitated the assessment of interactions among lichen species on individual trees. We
hypothesize that in natural systems evolution occurs in a community context involving interactions of complex networks
of interacting species (22, 23, 28, 32). If correct, we expect to find that network structure is genetically based, or,
in other words, plant genotypes will support different and heritable interaction networks. Applying a
probability-theory based network modeling approach, we constructed a set of interaction network models for the lichens
associated with individual trees. Using these models, we then examined the genetic basis of the structure of these
ecological networks via several network metrics that measures different aspects of network structure at the scale of
individual species (i.e., nodes) or the entire network observed on each tree genotype. In particular, we focus the
metric of centrality for individual species and centralization for whole networks, which measures how much a species is
connected in the network relative to other species. Based on previous community genetics theory, particularly the
community similarity rule (15), we hypothesize that trees will co-vary in functional phenotypic traits such as bark
roughness and chemical composition and trees of the same genotype will tend to have similar traits leading to
similarities in lichen network structure. This work is important because it provides a mechanistic basis for
understanding how community network theory is intimately associated with the evolutionary process and how human
alterations of the environment (e.g., climate change, invasive species, pollution) may have cascading, indirect effects
that alter network structure and evolution.
\end{document}
