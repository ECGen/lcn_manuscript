\documentclass[fleqn,12pt]{article}

\begin{document}

For Submission to Nature Ecology and Evolution

Elucidating the genetic basis to ecological network structure is
fundamental to understanding evolution in complex ecosystems. Although
previous work has demonstrated that genetic variation can influence
food webs and trophic chains, we are unaware of a study that
quantified the contribution of phenotypic variation to heritable
variation in network structure. To examine this, in a 20+ year old
common garden we observed epiphytic lichens associated with narrowleaf
cottonwood (\textit{Populus angustifolia}), a riparian ecosystem
foundation species. We constructed and conducted genetic analyses of
signed, weighted, directed lichen interaction networks on individual
trees. We found three primary results. First, genotype identity
significantly predicted lichen network similarity; i.e., replicates of
the same genotype supported more similar lichen networks than
different genotypes. Second, broad sense heritability estimates showed
that plant genotype explained network similarity ($H^2$ = 0.41),
degree ($H^2$ = 0.32) and centralization ($H^2$ = 0.33). Third, of
several tree phenotypic traits examined, bark roughness was both
heritable ($H^2$ = 0.32) and significantly predicted by lichen network
similarity ($R^2$ = 0.26). These results support a mechanistic,
genetic pathway from variation in a heritable tree trait to ecological
network structure and demonstrate that evolution can act at the
community level to influence not only abundances of organisms but also
interactions at the scale of entire networks. Given that network
structure can influence system-wide stability and resilience, our
findings have important implications for how evolution acts in
ecosystems.





\end{document}
