%% Basic Academic Journal Article Template
%% Author
%% John Hammersley
%% License
%% Creative Commons CC BY 4.0
%% Abstract
%% This is a basic journal article template which includes metadata fields for multiple authors, affiliations and keywords.

%% It is also set up to use the lineno package for line numbers; these can be turned on by adding the 'lineno' option to the documentclass command.

%% Tags
%% Find More Templates
%% Basic Academic Journal Article Template
%% © 2020 OverleafPrivacy and TermsSecurityContact UsAboutBlog
%% Overleaf on TwitterOverleaf on FacebookOverleaf on LinkedIn
\documentclass[fleqn,12pt]{olplainarticle}

% Use option lineno for line numbers 

\usepackage[utf8]{inputenc}
\usepackage{graphicx}

\title{Genotypic variation in a foundation tree results in heritable
  ecological network structure of an associated community}

\author[1,2]{Matthew K. Lau}
\author[2]{Louis J. Lamit}
\author[3]{Rikke R. Naesbourg}
\author[4]{Stuart R. Borrett}
\author[5]{Matthew A. Bowker}
\author[6]{Thomas G. Whitham}

\affil[1]{Department of Biological Sciences and Merriam-Powell Center
  for Environmental Research, Northern Arizona University, Flagstaff,
  AZ 86011, USA}
\affil[2]{Harvard Forest, Harvard University, 324 N Main St,
  Petersham, MA 01366, USA}
\affil[3]{Department of Biology, Syracuse University, 107 College
  Place Syracuse, NY 13244, USA}
\affil[4]{University of California Berkeley, Berkeley, CA, USA}
\affil[5]{Department of Biology and Marine Biology, University of
  North Carolina Wilmington, 601 South College Road, Wilmington, NC,
  28403, USA}
\affil[6]{School of Forestry, Northern Arizona University, Flagstaff,
  AZ 86011, USA}

\keywords{networks $|$ heritability $|$ community $|$ genetics $|$
  lichen $|$ cottonwood $|$ \textit{Populus} $|$ common garden}

\begin{abstract}

%% Currently: 250 words (Thu 05 Nov 2020 10:55:40 AM EST)

Biological evolution occurs in ecosystems whereby natural selection
defines the structure of ecological networks. Therefore, elucidating
the genetic basis to ecological network structure is fundamental to
understanding evolution. Although previous work has demonstrated that
genetic variation can influence food webs and trophic chains, we are
unaware of a study that quantified the contribution of phenotypic
variation to heritable variation in network structure. To examine
this, in a 20+ year common garden we observed nine epiphytic lichen
species associated with narrowleaf cottonwood (\textit{Populus
  angustifolia}), a riparian ecosystem foundation species. We
constructed and conducted genetic analyses of signed, weighted,
directed lichen interaction networks. We found three primary
results. First, genotype identity significantly predicted lichen
network similarity; i.e., replicates of the same genotype supported
more similar lichen networks than different genotypes. Second, broad
sense heritability estimates showed that plant genotype explained
network similarity ($H^2$ = 0.41), degree ($H^2$ = 0.32) and
centralization ($H^2$ = 0.33). Third, of several tree phenotypic
traits examined, bark roughness was both heritable ($H^2$ = 0.32) and
significantly correlated with lichen network similarity ($R^2$ =
0.26). These results support a mechanistic, genetic pathway from
variation in a heritable tree trait to ecological network structure
and demonstrate that evolution can act at the community level to
influence not only abundances of organisms but also interactions at
the scale of entire networks. Given that network structure has
determines system-wide stability and resilience, our findings have
important implications for how evolution acts in ecosystems.

\end{abstract}

\dates{This manuscript was compiled on \today}

\begin{document}

\flushbottom
\maketitle
\thispagestyle{empty}



\section*{Introduction}

Thanks for using Overleaf to write your article. Your introduction goes here! Some examples of commonly used commands and features are listed below, to help you get started.

\section*{Methods and Materials}

Guidelines can be included for standard research article sections, such as this one.

\section*{Some \LaTeX{} Examples}
\label{sec:examples}

Use section and subsection commands to organize your document. \LaTeX{} handles all the formatting and numbering automatically. Use ref and label commands for cross-references.

\subsection*{Figures and Tables}

Use the table and tabular commands for basic tables --- see Table~\ref{tab:widgets}, for example. You can upload a figure (JPEG, PNG or PDF) using the project menu. To include it in your document, use the includegraphics command as in the code for Figure~\ref{fig:view} below.

\begin{figure}[ht]
\centering
\includegraphics[width=0.7\linewidth]{frog}
\caption{An example image of a frog.}
\label{fig:view}
\end{figure}

\begin{table}[ht]
\centering
\begin{tabular}{l|r}
Item & Quantity \\\hline
Candles & 4 \\
Fork handles & ?  
\end{tabular}
\caption{\label{tab:widgets}An example table.}
\end{table}

\subsection*{Citations}

LaTeX formats citations and references automatically using the bibliography records in your .bib file, which you can edit via the project menu. Use the cite command for an inline citation, like \cite{lees2010theoretical}, and the citep command for a citation in parentheses \citep{lees2010theoretical}.

\subsection*{Mathematics}

\LaTeX{} is great at typesetting mathematics. Let $X_1, X_2, \ldots, X_n$ be a sequence of independent and identically distributed random variables with $\text{E}[X_i] = \mu$ and $\text{Var}[X_i] = \sigma^2 < \infty$, and let
$$S_n = \frac{X_1 + X_2 + \cdots + X_n}{n}
      = \frac{1}{n}\sum_{i}^{n} X_i$$
denote their mean. Then as $n$ approaches infinity, the random variables $\sqrt{n}(S_n - \mu)$ converge in distribution to a normal $\mathcal{N}(0, \sigma^2)$.

\subsection*{Lists}

You can make lists with automatic numbering \dots

\begin{enumerate}[noitemsep] 
\item Like this,
\item and like this.
\end{enumerate}
\dots or bullet points \dots
\begin{itemize}[noitemsep] 
\item Like this,
\item and like this.
\end{itemize}
\dots or with words and descriptions \dots
\begin{description}
\item[Word] Definition
\item[Concept] Explanation
\item[Idea] Text
\end{description}

\section*{Acknowledgments}

Additional information can be given in the template, such as to not include funder information in the acknowledgments section.

\bibliography{sample}

\end{document}
