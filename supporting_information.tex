\section{Supporting Information}

%% Authors should submit SI as a single separate PDF file, combining
%% all text, figures, tables, movie legends, and SI references.  PNAS
%% will publish SI uncomposed, as the authors have provided it.
%% Additional details can be found here:
%% \href{http://www.pnas.org/page/authors/journal-policies}{policy on
%% SI}.  For SI formatting instructions click
%% \href{https://www.pnascentral.org/cgi-bin/main.plex?form_type=display_auth_si_instructions}{here}.
%% The PNAS Overleaf SI template can be found
%% \href{https://www.overleaf.com/latex/templates/pnas-template-for-supplementary-information/wqfsfqwyjtsd}{here}.
%% Refer to the SI Appendix in the manuscript at an appropriate point
%% in the text. Number supporting figures and tables starting with S1,
%% S2, etc.

%% Authors who place detailed materials and methods in an SI Appendix
%% must provide sufficient detail in the main text methods to enable a
%% reader to follow the logic of the procedures and results and also
%% must reference the SI methods. If a paper is fundamentally a study
%% of a new method or technique, then the methods must be described
%% completely in the main text.

\subsection{Species level network analysis}


\textit{Candaleriella subdeflexa} was generally the most central
species (i.e. being the most highly connected) having the highest
average centrality (0.73), followed by \textit{Ca. holocarpa} (0.54)
and \textit{L. hageni} (0.40). The centralization of the remaining
species were \textit{R.}  sp. (0.18), \textit{X. galericulata} (0.14),
\textit{P. melanchra} (0.08), \textit{X. montana} (0.06) and
\textit{Ph. undulata} (0.02). \textit{Physcia adscendens} was
generally not connected to other species in the networks and had a
centralization score of zero.

